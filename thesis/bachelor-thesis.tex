\documentclass[times,specification,annotation]{style/itmo-student-thesis/itmo-student-thesis}

\usepackage{icomma}
\usepackage{tabularx}

\usepackage{adjustbox}

\usepackage{tikz}
\usetikzlibrary{arrows}

\usepackage{filecontents}
\begin{filecontents}{bachelor-thesis.bib}
% TODO
\end{filecontents}

\addbibresource{bachelor-thesis.bib}

\begin{document}

\studygroup{M34371}
\title{Автоматизация построения программных компонентов при подготовке задач по спортивному программированию}
\author{Назаров Георгий Дмитриевич}{Назаров Г.Д.}
\supervisor{Корнеев Георгий Александрович}{Корнеев Г.А.}{доцент, к.т.н.}{доцент факультета информационных технологий и программирования}
\publishyear{2023}
\startdate{10}{апреля}{2023}
\finishdate{15}{мая}{2023}

\addconsultant{Мирзаянов М.Р.}{без степени, без звания}

\secretary{Штумпф С.А.}

\technicalspec{
    Требуется разработать систему, автоматизирующую процесс построения программных компонентов при подготовке (разработке) задач по спортивному программированию, в том числе:

    1. Спроектировать специальный язык описания входных и выходных данных, требуемых в задаче по спортивному программированию.

    2. Реализовать систему кодогенерации, которая, на основании описания входных и выходных данных, будет генерировать исходный код программных компонентов задачи, в том числе:

    2.1. Программы, проверяющей ответ участника (чекера);

    2.2. Программы, проверяющей корректность входных данных (валидатора);

    2.3. Компонентов ввода-вывода пользовательских решений задачи (грейдеров).

    3. Внедрить полученную систему в сервис подготовки задач по спортивному программированию «Polygon».
}

\plannedcontents{TODO}

\plannedsources{\begin{enumerate}
    \item TODO
\end{enumerate}}

\researchaim{TODO}

\researchtargets{\begin{enumerate}
    \item TODO
\end{enumerate}}

\addadvancedsoftware{TODO}{TODO}

\researchsummary{TODO}

\researchfunding{TODO}

\researchpublications{TODO}

\maketitle{Бакалавр}

\tableofcontents

\startprefacepage

Соревнования по спортивному программированию проходят с использованием автоматизированных систем тестирования, например, PCMS2, Ejudge, Яндекс.Контест, Codeforces. Участники отсылают на проверку исходный код программы-решения, которая автоматически проверяется на наборе тестов. 

Для того, чтобы задачу по спортивному программированию можно было использовать в автоматизированной тестирующей системе, авторы задачи должны предварительно подготовить <<пакет>> задачи, состоящий из условий задачи, тестов и набора программ, участвующих в процессе тестирования. Подготовка каждой задачи~--- трудоемкий ручной процесс, легко подверженный ошибкам.

В работе реализована система, позволяющая частично автоматизировать процесс подготовки задачи путем генерации некоторых  программных компонентов задачи.

% TODO: Краткое описание структуры работы вида “В Главе X рассмотрен вопрос...”

\chapter{Обзор предметной области}

% \startrelatedwork
% TODO
% \finishrelatedwork

В данной главе приводятся описания существующего процесса подготовки задач по спортивному программированию и процесса участия в соревнованиях по спортивному программированию с точки зрения участника. В дальнейшей работе будут использоваться термины и определения, вводимые в этой главе.

\section{Соревнования с точки зрения участников}

На соревнованиях по спортивному программированию участникам предлагается к решению упорядоченный набор задач. За ограниченное количество времени участники должны решить как можно больше задач и как можно точнее (определение <<точности>> зависит от правил конкретного соревнования).

К каждой задаче прилагается условие задачи~--- текстовое человекочитаемое описание требований к решению участника. Решением участника является исходный код на одном из доступных в автоматизированной системе тестирования языков программирования. Решение задачи может содержать, в зависимости от правил соревнования, как весь исходный код программы-решения, так и его часть~--- реализацию заданного в условии задачи интерфейса или функции с необходимой сигнатурой.

После проверки решения, участнику сообщается вердикт проверки~--- строка с информацией о корректности решения, возможно, содержащая дополнительную информацию, например, номер теста, на котором решение участника выдало неправильный ответ.

\section{Процесс тестирования}

После получения решения участника автоматизированная тестирующая система выполняет его обработку, состоящую из следующих основных шагов:

\begin{enumerate}
    \item Компиляция решения участника;
    \item Компоновка (линковка) решения участника с компонентами задачи (при необходимости);
    \item Генерация наборов входных данных (для генерируемых тестов);
    \item Валидация входных данных на соответствие условию задачи;
    \item Запуск исполняемого файла с решением на наборе тестов (входных данных);
    \item Запуск программы проверки ответа участника на каждом из тестов и ответов программы-решения;
\end{enumerate}

В зависимости от правил соревнования шагов может быть больше. Об ошибке на любом из шагов сообщается участнику.

\section{Подготовка задач}

Чтобы задачу можно было использовать в автоматизированной тестирующей системе, авторы задачи должны подготовить следующие компоненты задачи:

\begin{enumerate}
    \item Условия задачи, возможно, на нескольких языках;
    \item Валидатор (от англ. \textit{validator})~--- программу, принимающую на вход произвольный текстовый файл и выдающую вердикт, может ли быть этот файл использован в качестве теста в конкретной задачи. Валидатор должен проверять структуру файла и соответствие значений переменных ограничениям из условия;
    \item Чекер (от англ. \textit{checker})~--- программу, принимающую на вход тест, ответ программы участника и ответ жюри (авторского решения), и выдающая вердикт о корректности ответа участника. Чекер дожен проверять выполнение в ответе ограничений из условия задачи.
    \item Тесты (входные данные)~--- текстовые файлы, которые будут подаваться на вход решению участника. Могут быть заданы вручную или генерироваться во время тестирования;
    \item Генераторы~--- программы, принимаюцие в аргументах командной строки параметры, на основании которых генерируют (выводят) тест или набор тестов;
    \item Решение жюри (авторское решение)~--- исходный код эталонного решения задачи. Используется для получения эталонных ответов при запуске чекера;
\end{enumerate}

На некоторых соревнованиях от участников не требуется полная реализация решения, а только лишь его части~--- функции с заданной сигнатурой или интерфейса. (Подобные правила применяются, например, на IOI~--- международной олимпиаде по информатике). В таком случае авторам задачи так же необходимо подготовить грейдеры (от англ. \textit{grader}, <<оценщик>>)~--- часть исходного кода решения, которая будет скомпонована (слинкована) с решением участника. Исходный код грейдеров должен обязательно содержать точку входа программы и реализацию ввода-вывода, специфичную для задачи. Опционально, грейдеры могут выполнять дополнительные действия перед передачей управления коду участника или после окончания выполнения кода участника. Грейдеры должны быть написаны на каждом из доступных участнику языков программирования.

Рассмотрим подробнее, что представляют собой тесты (входные данные) и выходные данные в задачах по спортивному программированию. Как было сказано выше, это текстовые файлы с определенной автором задачи структурой. В условии задачи описываются переменные и ограничения на них~--- тесты, в свою очередь, состоят из значений этих переменных, причем значения разделены пробелами и переводами строк в порядке, определенном автором задачи (и указанном в условии). Выходные данные так же состоят из значений переменных, разделенных в установленном порядке пробелами и переводами строк.

\section{Примеры программных компонентов задачи}

Рассмотрим подробнее программные компоненты на примере задачи <<A~+~B>>.

На листинге~\ref{a-plus-b-validator} приведен исходный код валидатора. В нем считывается два целочисленных значения в интервале от $-1000$ до $1000$, разделенных одним пробелом. Затем ожидается перевод строки и конец файла теста.

\begin{lstlisting}[float=!h,caption={Пример валидатора},label={a-plus-b-validator},language=c++]
#include "testlib.h"

int main(int argc, char *argv[])
{
    registerValidation(argc, argv);
    inf.readInt(-1000, 1000, "a");
    inf.readSpace();
    inf.readInt(-1000, 1000, "b");
    inf.readEoln();
    inf.readEof();
    return 0;
}
\end{lstlisting}

На листинге~\ref{a-plus-b-checker} приведен исходный код чекера. В нем считывается одно целочисленное значение в интервале от $-2000$ до $2000$ два раза: сначала ответ участника, затем ответ жюри (вывод авторского решения). Если ответы совпали, участник получает вердикт <<Правильный ответ>>, в противном случае <<Неправильный ответ>>, на соответствующем тесте.

\begin{lstlisting}[float=!h,caption={Пример чекера},label={a-plus-b-checker},language=c++]
#include "testlib.h"

int readAns(InStream& stream)
{
    return stream.readInt(-2000, 2000, "sum");
}

int main(int argc, char *argv[])
{
    registerTestlibCmd(argc, argv);
    int participant = readAns(ouf);
    int jury = readAns(ans);
    if (participant != jury)
        quitf(_wa, "Wrong answer");
    else
        quitf(_ok, "Ok");
}
\end{lstlisting}

Тесты могут быть сгенерированны программно. На листинге~\ref{a-plus-b-generator} приведен исходный код генератора случайных тестов. Принимает на вход единственный аргумент командной строки~--- ограничение на абсолютную величину генерируемых значений. Два случайных значения выводятся в стандартный поток вывода через пробел, затем следует перевод строки.

\begin{lstlisting}[float=!h,caption={Пример генератора},label={a-plus-b-generator},language=c++]
#include <iostream>
#include "testlib.h"
 
int main(int argc, char* argv[]) {
    registerGen(argc, argv, 1);
 
    int N = opt<int>(1);
    std::cout << rnd.next(-N, N) << ' ';
    std::cout << rnd.next(-N, N) << std::endl;
    return 0;
}
\end{lstlisting}

Грейдеры являются написанной жюри (автором задачи) частью исходного кода, который будет скомпонован с реализацией участника для получения полного исполняемого файла с решением. Грейдеры должны быть написаны на всех, доступных для использования участниками, языках программирования. 

На листинге~\ref{a-plus-b-grader-cpp} приведен исходный код грейдера для языка программирования C++. Грейдеры для C++ обычно состоят из двух файлов: заголовочного и главного. Заголовочный файл, импортируемый кодом участника, содержит определение требуемых к реализации функций и функций, написанных жюри, доступных для использования участником. Главный файл содержит точку входа (функцию \textit{main()}, с которой будет начато исполнение) и код, считывающий входные данные из стандартного потока ввода и выводящий ответ, полученный вызовом пользовательской реализации, в стандартный поток вывода.

\begin{lstlisting}[float=!h,caption={Пример грейдера для языка C++},label={a-plus-b-grader-cpp},language=c++]
// aplusb.h

int sum_ab(int a, int b);

// grader.cpp

#include <iostream>
#include "aplusb.h"
 
int main() {
    int a, b;
    std::cin >> a >> b;
    std::cout << sum_ab(a, b) << std::endl;
    return 0;
}
\end{lstlisting}

На листинге~\ref{a-plus-b-grader-py} приведен исходный код грейдера для языка программирования Python. Так как Python~--- интерпретируемый язык программирования, понятие линковки для него не определено, поэтому грейдер напрямую импортирует и вызывает решение участника. Блок кода в середине содержит проверку версии языка, чтобы грейдер мог работать как на Python 2, так и на Python 3.

\begin{lstlisting}[float=!h,caption={Пример грейдера для языка Python},label={a-plus-b-grader-py},language=python]
import solution
import sys

if sys.version_info[0] < 3:
    _input = raw_input
else:
    _input = input
 
a, b = map(int, _input().split())
print(solution.sum_ab(a, b))
\end{lstlisting}

\section{Цели и задачи ВКР}

В предыдущих пунктах показано, что для подготовки даже простых задач по спортивному программированию авторам задачи необходимо проделать существенную работу, написав исходный код для, как минимум, трёх программных компонентов~--- валидатора, чекера и генератора, при этом генераторов может быть много, для того, чтобы сгенерировать как можно более разнообразные тесты. При этом, в случае, если задача подразумевает наличие грейдеров, объем исходного кода, необходимого к написанию, становится пропорциональным количеству поддерживаемых тестирующей системой языков программирования.

% Задание на ВКР

Основная цель данной выпускной квалификационной работы~--- разработка системы, автоматизирующей процесс реализации программных компонентов при подготовке задач по спортивному программированию.

В ходе консультаций с авторами задач по спортивному программированию были выделены следующие задачи, решаемые в данной выпускной квалификационной работе:

\begin{enumerate}
    \item Необходимо спроектировать специальный язык описания входных и выходных данных, специфичных для задачи.
    \item Реализовать систему кодогенерации, которая будет принимать на вход описание входных и выходных данных, на основании которой будет частично сгенерирован исходный код валидатора, чекера и грейдеров.
    \item Внедрить реализованную систему в сервис подготовки задач по спортивному программированию <<Polygon>>.
\end{enumerate}

% \section{Используемые в работе определения и инструменты}

% \subsection{Определения}

% \begin{definition}
%     Формальный язык~--- множество допустимых строк (последовательностей символов) над некоторым алфавитом.
% \end{definition}

% \begin{definition}
%     Алфавит~--- конечное непустое множество символов.
% \end{definition}

% \begin{definition}
%     Формальная грамматика~--- способ задания формального язы
% \end{definition}

% \subsection{Инструменты}

\chapter{Проектирование языка описания входных и выходных данных и реализация кодогенерации}

В этой главе формулируются требования к языку описания входных и выходных данных, описывается полученный язык описания и реализация системы кодогенерации.

\section{Требования к языку описания}

В ходе консультаций с авторами задач по спортивному программированию были выделены следующие требования к возможностям языка описания:

\begin{enumerate}
    \item Должна быть возможность указывать название для каждой переменной. Это необходимо для генерации человекочитаемых сообщений об ошибках в валидаторе и чекере.
    \item Должна быть возможность указывать тип переменных, для корректной реализации ввода-вывода.
    \item Должна быть возможность описывать структуры~--- упорядоченный набор семантически связных переменнных.
    \item Должна быть возможность указывать места перевода строк.
    \item Должна быть возможность задавать ограничения на переменные~--- для проверки корректности данных.
\end{enumerate}

Отдельно стоит отметить одно неформальное требование: язык описания должен иметь <<низкий порог входа>>. Иными словами, <<простые>> входные и выходные данные должны с его помощью описываться <<простыми>> конструкциями.

Так же для удобства последующей реализации было принято решение ограничить возможный язык описания классом $LL(1)$ контекстно-свободных грамматик.

\section{Язык описания входных и выходных данных}

Входные и выходные данных представляют собой текстовые файл, содержимое которых удовлетворяет описанным в условии задачи ограничениям и формату. Разработанный язык описания представляет способ текстового машиночитаемого описания структуры входных и выходных данных, понятного человеку. Для построения лексического и синтаксического анализаторов языка описания был использован генератор нисходящих анализаторов для формальных языков \textit{ANTLR 4}. Полная грамматика языка описания представлена в Приложении~\ref{appendix-antlr-grammar}.

Файл описания входных и выходных данных состоит из набора \textit{структур}~--- объявлений семантически объединенных переменных, идущих во входных или выходных данных в указаном в описании порядке. Переменные внутри структуры могут представлять собой значения, массивы или другие структуры.

Каждая структура опционально может иметь некоторое, возможно нулевое, количество \textit{параметров}~--- переменных или выражений, которые необходимы для ввода-вывода соответствующей структуры. Структуры могут содержать переменные, условные операторы и модификаторы ввода-вывода. Условные операторы используются для изменения набора перменных в структуре в зависимости от какого-либо условия~--- выражения, вычислимого из переменных с известным значением (то есть тех, которые будут к моменту ввода-вывода текущей структуры уже содержались в вводе или выводе). Модификаторы ввода-вывода используются для изменения разделителей между переменными и указания необходимого форматирования во входных и выходных данных.

Объявление переменной содержит название соответствующей переменной, её тип и, опционально, ограничение на эту переменную, представленное выражением, записанном на языке выражений, который будет рассмотрнен позднее. Выражения в ограничениях должны иметь логический тип.

Подробное описание языковых конструкций будет рассмотрено позднее в данной работе.

\subsection{Лексика языка описания}

Листинг~\ref{appendix-lexer-src} содержит определения используемых в языке типов токенов (лексем)~--- неделимых синтаксических конструкций языка. Рассмотрим типы токенов подробнее.

Типы токенов \texttt{LINE\_COMMENT}, \texttt{COMMENT} и \texttt{WS} отвечают за комментарии и пробельные символы. Такие токены должны впоследствии игнорироваться синтаксическим анализатором, что указано директивой \texttt{-> skip} после них. Однострочные комментарии начинаются с символа <<\texttt{\#}>> и продолжаются до конца строки. Блоковые комментарии обрамляются последовательностями символов <<\texttt{/*}>> и <<\texttt{*/}>>.

Типы токенов \texttt{CHAR} и \texttt{STRING} отвечают за символьные и строковые константы соответственно.

Типы токенов \texttt{TRUE} и \texttt{FALSE} отвечают за логические константы <<истина>> и <<ложь>> соответственно.

Типы токенов \texttt{LOGICAL\_AND}, \texttt{LOGICAL\_OR} и \texttt{LOGICAL\_NOT} отвечают за операторы логической конъюнкции, дизъюнкции и отрицания. 

Типы токенов \texttt{BITWISE\_AND}, \texttt{BITWISE\_XOR} и \texttt{BITWISE\_NOT} отвечают за операции <<Побитовое И>>, <<Побитовое исключающее ИЛИ>> и <<Инверсия битов>>. Тип токена \texttt{PIPE} является неоднозначным, но, в том числе, отвечает за операцию <<Побитовое ИЛИ>>.

Типы токенов \texttt{EQUALS}, \texttt{NOT\_EQUALS}, \texttt{LESS\_EQUAL}, \texttt{GREATER\_EQUAL}, \texttt{LESS} и \texttt{GREATER} отвечают за операторы сравнения: равенство, неравенство, <<меньше или равно>>, <<больше или равно>>, <<меньше>> и <<больше>> соответственно.

Затем следует блок типов токенов, отвечающих за арифметические операции: \texttt{PLUS}, \texttt{MINUS}, \texttt{POW}, \texttt{MULTIPLICATION}, \texttt{DIVISION}, \texttt{MODULO}~--- сложение, вычитание, возведение в степень, умножение, деление и взятие остатка от деления (взятие по модулю) соответственно.

Тип токена \texttt{QUESTION} используется в тернарном условном операторе, как и многозначный \texttt{COLON}.

Тип токена \texttt{ASSIGN} используется в некоторых синтаксических конструкциях описания входных и выходных данных, семантически схожих с операцией присваивания значения переменной.

Тип токена \texttt{COMMA} используется в качестве разделителя аргументов функций и в семантически похожих конструкциях.

Тип токена \texttt{SEMICOLON} используется в качестве разделителя списка объявляемых переменных.

Типы токенов \texttt{LBRACE} и \texttt{RBRACE} используются для обозначения границ логического блока.

Типы токенов \texttt{LBRACKET} и \texttt{RBRACKET} используются в качестве обозначения индексации массивов и при определении массивов.

Типы токенов \texttt{LPAR} и \texttt{RPAR} используются для обозначения границ списка аргументов функций.

Затем следует блок типов токенов, являющихся зарезервированными словами языка описания входных и выходных данных. Это \texttt{SEP}, \texttt{EOLN\_MODIFIER}, \texttt{IF}, \texttt{ARRAY}, \texttt{OF} и \texttt{else}. Типы \texttt{IF} и \texttt{else} используются в условном операторе. Типы \texttt{SEP} и \texttt{EOLN\_MODIFIER} используются при указании разделителя между переменными во входных и выходных данных. Типы \texttt{ARRAY} и \texttt{OF} используются при объявлении массивов неименнованных структур.

Затем следует составное определение типа токена \texttt{NUM\_VALUE}. Токены этого типа могут содержать целочисленные значения или числа с плавающей точкой (в том числе в экспоненциальном виде). Используя суффиксы можно явно указать тип, в который будет транслироваться токен при кодогенерации.

Тип токена \texttt{NAME} отвечает за идентификаторы. Для упрощения последующей кодогенерации, идентификаторы могут содержать только буквы латинского алфавита, арабские цифры и символы <<\texttt{\_}>>.

Как можно видеть из листинга~\ref{appendix-lexer-src}, почти все операторы позаимствованы из языка программирования C++, за исключением оператора возведения в степень, так как соответствующий оператор в C++ отсутствует. Это сделано чтобы авторы задач могли писать выражения на языке максимально приближенном к С++, на котором подготавливается подавляющее большинство задач по спортивному программированию. Далее так же будет показано, что приоритеты операторов так же идентичны таковым в языке програмирования C++.

\subsection{Синтаксис языка описания}

Листинг~\ref{appendix-grammar-src} содержит формальную грамматику языка описания входных и выходных данных в формате ANTLR. Стартовым символом грамматики является нетерминал \texttt{ioMarkup}, раскрывающийся в список нетерминалов \texttt{namedStruct}~--- список определений структур.

Каждое определение структуры начинается с токена с типом \texttt{NAME}~--- названия соответствующей структуры. Названия структур должны быть уникальными и не совпадать с названиями встроенных типов. Затем следует необязательные параметры структуры (нетерминал \texttt{namedStructParameters}) и после них описание структуры (нетерминал \texttt{struct}).

Параметры структуры состоят из списка объявлений параметров (нетерминалы \texttt{parameterDeclaration}), разделенных запятыми, обрамленного круглыми скобками. Каждое объявление параметра состоит из имени параметра (\texttt{NAME}) и названия типа этого параметра, разделенных символом двоеточия.

Описание структуры состоит из списка элементов описания (нетерминалы \texttt{structItem}), обрамленного фигурными скобками. Элементами описания могут быть условные операторы (нетерминал \texttt{conditionalAlternative}), объявления переменных (нетерминал \texttt{variableDeclaration}) или модификаторы ввода-вывода (нетерминал \texttt{ioModifier}).

Объявления переменных состоят из названия переменной, затем следует символ двоеточия, после него пишется тип переменной (нетерминал \texttt{variableType}). Затем, опционально, следует символ вертикальной черты и ограничение на переменную, записанное на языке выражений (нетерминал \texttt{plExpression} из листинга~\ref{appendix-grammar-pl-src}). Завершается описание переменной символом точки с запятой.

Тип переменной задается следующим образом. Сначала записывается название типа (предопределенное или название объявленной автором задачи структуры), затем, опционально, следуют параметры структуры (нетерминал \texttt{namedTypeParameters}), состоящие из списка выражений, после чего, опционально, следуют одно или несколько деклараций массива (нетерминалы \texttt{arrayParameters}).

Декларации массива меняют тип переменной с одиночного на массив соответствующего типа. Декларация массива состоит из объявления переменной итерирования с указанием начального и конечного значений итерирования (нетерминал \texttt{arrayIterationRange}). Затем, через запятую, указывается разделитель переменных в массиве~--- строка, состоящая из пробелов или экранированных переводов строк (нетерминал \texttt{sepValue}).

Поддержана возможность определения массивов неименнованных структур (массивов кортежей с именнованными элементами). Для этого в качестве типа переменной указывается конструкция <<\texttt{array ... of}>> (нетерминал \texttt{arrayOfUnnamedStruct}), после которой следует описание структуры. Вместо двоеточия пишутся декларации массива.

\subsection{Синтаксис выражений}

Для записи выражений в декларациях массивов и ограничениях на переменные используется специальный синтаксис выражений, представленный ANTLR грамматикой в листинге~\ref{appendix-grammar-pl-src}. Стартовым нетерминалом является \texttt{plExpression}.

Нетерминал \texttt{plValue} отвечает за запись константных значений. Нетерминал \texttt{plVarBinding} отвечает за использование значения переменной в выражении. Операторы языка выражений и остальные нетерминалы представлены в таблице~\ref{pl-operators-priority}. Стартовым нетерминалом грамматики является \texttt{plExpression}.

Выражения записываются в виде, максимально приближенном к языку программирования C++, как к наиболее популярному языку программирования, использующегося при подготовке задач.

\subsection{Система типов}

Язык описания входных и выходных данных имеет встроенные целочисленные, строковые, символьные, логические типы и тип вещественных чисел с плавающей точкой. Полный список встроенных типов приведен в таблице~\ref{predefined-types}.

\begin{table}[!h]
\caption{Встроенные типы данных}\label{predefined-types}
\centering
\begin{tabular}{|*{3}{c|}}\hline
Тип & Описание & Пример\\\hline
\texttt{bool} & Логический тип & \texttt{true} \\\hline
\texttt{char} & Символьный тип & \texttt{'x'}\\\hline
\texttt{int32} & $32$-х битный целочисленный знаковый тип & \texttt{123} \\\hline
\texttt{uint32} & $32$-х битный целочисленный беззнаковый тип & \texttt{123u} \\\hline
\texttt{int64} & $64$-х битный целочисленный знаковый тип & \texttt{123l} \\\hline
\texttt{uint64} & $64$-х битный целочисленный беззнаковый тип & \texttt{123ul} \\\hline
\texttt{float32} & $32$-х битный тип чисел с плавающей точкой & \texttt{12.3f} \\\hline
\texttt{float64} & $64$-х битный тип чисел с плавающей точкой & \texttt{12.3} \\\hline
\texttt{string} & Строковый тип (массив символов) & \texttt{"abc"} \\\hline
\end{tabular}
\end{table}

Язык описания выходных и выходных данных поддерживает структуры данных~--- объявленные пользователем составные типы, состоящие из упорядоченного набора переменных других, возможно составных, типов. Пример структуры данных приведен на листинге~\ref{struct-example}.

\begin{lstlisting}[
    caption={Пример структуры},
    keywordstyle=\color{black},
    commentstyle=\color{black},
    stringstyle=\color{black},
    label={struct-example}
]
pair {
    int32 a;
    int32 b;
}
\end{lstlisting}

К любому типу переменной при объявлении может быть дописан суффикс, являющийся декларацией массива. Такая конструкция объявляет массив указанного типа.

\section{Обработка языка описания}

Был реализован сервис, принимающий на вход описание входных и выходных данных на описанном выше языке. Обработка происходит в несколько этапов:

\begin{enumerate}
    \item Лексический разбор,
    \item Синтаксический разбор,
    \item Поиск определений структур,
    \item Разрешение имен и типов, получение абстрактных синтаксических деревьев выражений и структур,
    \item Генерация кода.
\end{enumerate}

Лексический разбор~--- процесс перевода текстового файла, поданного на вход, в список токенов. Синтаксический разбор~--- процесс перевода списка токенов в дерево разбора. Во время синтаксического разбора так же происходит проверка синтаксической корректности последовательности токенов. Эти два этапа выполняются при помощи сгенерированного ANTLR анализатора.

На следующем этапе выполняется поиск и регистрация имен структур данных, определенных пользователем. Это необходимо для того, чтобы впоследствии, при получении абстрактных синтаксических деревьев структур, знать полный набор именнованных типов, существующих в описании. Этот этап выполняется частичным обходом дерева разбора.

Затем выполняется основной этап обработки, включающий в себя получение абстрактных синтаксических деревьев для выражений и структур, а так же разрешение имен переменных и типов. Этот этап выполняется обходом дерева разбора. Во время обхода проверяется семантическая корректность, существование используемых типов (в том числе структур), существование переменных, ссылки на которых используются. Во время обхода поддерживается информация о текущей области видимости переменных. Для упрощения реализации, выражения (длины массивов и ограничения) обходятся отдельно.

Последним этапом является непосредственно генерация целевого кода чекера, валидатора и грейдеров. Для каждого генерируемого программного компонента задачи поодередно выполняется генерация определений структур, объявленных автором в файле описания входных и выходных данных, генерация кода ввода и вывода.

\startconclusionpage

TODO

\printmainbibliography

\appendix

\chapter{Грамматика языка описания}\label{appendix-antlr-grammar}

\begin{lstlisting}[
    caption={Определения токенов в формате ANTLR 4},
    keywordstyle=\color{black},
    commentstyle=\color{black},
    stringstyle=\color{black},
    label={appendix-lexer-src}
]
LINE_COMMENT: '#' .*? '\r'? ('\n' | EOF) -> skip;
COMMENT: '/*' .*? '*/' -> skip;

fragment ESCAPED_CHAR: '\\n';

CHAR: '\'' ((~['\\\r\n]) | ESCAPED_CHAR) '\'';
STRING: '"' ((~["\\\r\n]) | ESCAPED_CHAR)* '"';
TRUE: 'true';
FALSE: 'false';
LOGICAL_AND: '&&';
LOGICAL_OR: '||';
LOGICAL_NOT: '!';
BITWISE_AND: '&';
BITWISE_XOR: '^';
PIPE: '|';
BITWISE_NOT: '~';
EQUALS: '==';
NOT_EQUALS: '!=';
LESS_EQUAL: '<=';
GREATER_EQUAL: '>=';
LESS: '<';
GREATER: '>';
PLUS: '+';
MINUS: '-';
POW: '**';
MULTIPLICATION: '*';
DIVISION: '/';
MODULO: '%';
DOT: '.';
COLON: ':';
QUESTION: '?';
ASSIGN: '=';
COMMA: ',';
SEMICOLON: ';';
LBRACE: '{';
RBRACE: '}';
LBRACKET: '[';
RBRACKET: ']';
LPAR: '(';
RPAR: ')';
SEP: 'sep';
EOLN_MODIFIER: 'eoln';
IF: 'if';
ARRAY: 'array';
OF: 'of';
ELSE: 'else';

fragment U_NUM_LITERAL_MODIFIER: 'u' | 'U';
fragment L_NUM_LITERAL_MODIFIER: 'l' | 'L';
fragment F_NUM_LITERAL_MODIFIER: 'f' | 'F';
fragment EXPONENT: 'e' | 'E';
fragment DIGITS: [0-9][0-9'_]*;

fragment INTEGER_SUFFIX: U_NUM_LITERAL_MODIFIER? L_NUM_LITERAL_MODIFIER?;
fragment FLOAT_SUFFIX: ('.' DIGITS)? (EXPONENT DIGITS)? F_NUM_LITERAL_MODIFIER?;

NUM_VALUE: DIGITS INTEGER_SUFFIX FLOAT_SUFFIX;

NAME: [a-zA-Z_][a-zA-Z0-9_]*;
WS: [ \t\r\n]+ -> skip;
\end{lstlisting}

\begin{lstlisting}[
    caption={Грамматика языка описания в формате ANTLR 4},
    keywordstyle=\color{black},
    commentstyle=\color{black},
    stringstyle=\color{black},
    label={appendix-grammar-src}
]
ioMarkup: (namedStruct)* EOF;
namedStruct: NAME namedStructParameters? struct;
namedStructParameters: LPAR (parameterDeclaration (COMMA parameterDeclaration)*)? RPAR;
parameterDeclaration: NAME COLON namedType;
struct: LBRACE structItem* RBRACE;
structItem: conditionalAlternative | variableDeclaration | ioModifier;
conditionalAlternative: IF LPAR plExpression RPAR struct (ELSE struct)?;
ioModifier: EOLN_MODIFIER SEMICOLON;
variableDeclaration: NAME COLON variableType variableConstraint? SEMICOLON;
variableType: arrayOfUnnamedStruct | (namedType namedTypeParameters? arrayParameters*);
arrayOfUnnamedStruct: ARRAY arrayParameters+ OF struct;
variableConstraint: PIPE plExpression;
namedType: NAME;
namedTypeParameters: LPAR (plExpression (COMMA plExpression)*)? RPAR;
arrayParameters: LBRACKET arrayIterationRange (COMMA SEP ASSIGN sepValue) RBRACKET;
arrayIterationRange: NAME ASSIGN plExpression DOT DOT plExpression;
sepValue: STRING | CHAR;
\end{lstlisting}

\begin{lstlisting}[
    caption={Грамматика языка выражений в формате ANTLR 4},
    keywordstyle=\color{black},
    commentstyle=\color{black},
    stringstyle=\color{black},
    label={appendix-grammar-pl-src}
]
plExpression: plImplLogicalOr (QUESTION plImplLogicalOr COLON plImplLogicalOr)?;

plImplLogicalOr: plImplLogicalAnd (LOGICAL_OR plImplLogicalAnd)*;
plImplLogicalAnd: plImplBitwiseOr (LOGICAL_AND plImplBitwiseOr)*;
plImplBitwiseOr: plImplBitwiseXor (PIPE plImplBitwiseXor)*;
plImplBitwiseXor: plImplBitwiseAnd (BITWISE_XOR plImplBitwiseAnd)*;
plImplBitwiseAnd: plImplEquality (BITWISE_AND plImplEquality)*;
plImplEquality: plImplRelational (plImplEqualityOp plImplRelational)*;
plImplRelational: plImplBitwiseShift (plImplRelationalOp plImplBitwiseShift)?;
plImplBitwiseShift: plImplAdditiveBinary (plImplBitwiseShiftOp plImplAdditiveBinary)?;
plImplAdditiveBinary: plImplMultiplicativeBinary (plImplAdditiveOp plImplMultiplicativeBinary)*;
plImplMultiplicativeBinary: plImplPowBinary (plImplMultiplicativeBinaryOp plImplPowBinary)*;
plImplPowBinary: plImplPrefixUnary (POW plImplPrefixUnary)*;
plImplPrefixUnary: plImplPrefixUnaryOp* plImplHighestPriority;

plImplHighestPriority: ((LPAR plExpression RPAR) | plFunctionCall | plValue | plVarBinding) plSubscript*;

plImplEqualityOp: EQUALS | NOT_EQUALS;
plImplRelationalOp: GREATER | LESS | GREATER_EQUAL | LESS_EQUAL;
plImplAdditiveOp: PLUS | MINUS;
plImplMultiplicativeBinaryOp: MULTIPLICATION | DIVISION | MODULO;
plImplPrefixUnaryOp: LOGICAL_NOT | BITWISE_NOT | MINUS;
plImplBitwiseShiftOp: LESS LESS | GREATER GREATER;

plFunctionCall: NAME LPAR plImplFunctionArgs? RPAR;
plImplFunctionArgs: plExpression (COMMA plExpression)*;

plVarBinding: NAME;

plSubscript: LBRACKET plExpression RBRACKET;

plValue: plNumValue | plBoolValue | plCharValue | plStringValue;
plBoolValue: TRUE | FALSE;
plCharValue: CHAR;
plStringValue: STRING;
plNumValue: NUM_VALUE;
\end{lstlisting}

\chapter{Операторы языка выражений}\label{appendix-pl-operators-priority}

\begin{table}[!h]
\caption{Операторы языка выражений в порядке уменьшения приоритета}\label{pl-operators-priority}
\centering
\begin{adjustbox}{angle=-90}
\begin{tabular}{|*{4}{c|}}\hline
Приоритет & Нетерминал & Оператор & Описание\\\hline
1  & \texttt{plImplHighestPriority} & \texttt{f(x)} & Вызов функции \\\hline
1  & \texttt{plImplHighestPriority} & \texttt{x[y]} & Взятие по индексу \\\hline
2  & \texttt{plImplPrefixUnary} & \texttt{!x} & Логическое отрицание \\\hline
2  & \texttt{plImplPrefixUnary} & \texttt{~x} & Инверсия битов \\\hline
2  & \texttt{plImplPrefixUnary} & \texttt{-x} & Унарный минус \\\hline
3  & \texttt{plImplPowBinary} & \texttt{x ** y} & Возведение в степень \\\hline
4  & \texttt{plImplMultiplicativeBinary} & \texttt{x * y} & Умножение \\\hline
4  & \texttt{plImplMultiplicativeBinary} & \texttt{x / y} & Деление \\\hline
4  & \texttt{plImplMultiplicativeBinary} & \texttt{x \% y} & Остаток от деления \\\hline
5  & \texttt{plImplAdditiveBinary} & \texttt{x + y} & Сложение \\\hline
5  & \texttt{plImplAdditiveBinary} & \texttt{x - y} & Вычитание \\\hline
6  & \texttt{plImplBitwiseShift} & \texttt{x >{>} y} & Побитовый сдвиг вправо \\\hline
6  & \texttt{plImplBitwiseShift} & \texttt{x <{<} y} & Побитовый сдвиг влево \\\hline
7  & \texttt{plImplRelationalOp} & \texttt{>}, \texttt{<}, \texttt{>=}, \texttt{<=} & Сравнение \\\hline
8  & \texttt{plImplEquality} & \texttt{x == y} & Равенство \\\hline
8  & \texttt{plImplEquality} & \texttt{x != y} & Неравенство \\\hline
9  & \texttt{plImplBitwiseAnd} & \texttt{x \& y} & Побитовое <<И>> \\\hline
10  & \texttt{plImplBitwiseXor}  & \texttt{x \textasciicircum~y} & Побитовое исключающее <<ИЛИ>> \\\hline
11  & \texttt{plImplBitwiseOr}  & \texttt{x | y} & Побитовое <<ИЛИ>> \\\hline
12  & \texttt{plImplLogicalAnd}  & \texttt{x \&\& y} & Конъюнкция \\\hline
13  & \texttt{plImplLogicalOr}  & \texttt{x || y} & Дизъюнкция \\\hline
14  & \texttt{plExpression}  & \texttt{x ? y : x} & Тернарный условный оператор \\\hline
\end{tabular}
\end{adjustbox}
\end{table}

\end{document}
